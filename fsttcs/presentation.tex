% Created 2025-12-11 Thu 15:11
% Intended LaTeX compiler: pdflatex
\documentclass[presentation]{beamer}
\usepackage[utf8]{inputenc}
\usepackage[T1]{fontenc}
\usepackage{graphicx}
\usepackage{longtable}
\usepackage{wrapfig}
\usepackage{rotating}
\usepackage[normalem]{ulem}
\usepackage{amsmath}
\usepackage{amssymb}
\usepackage{capt-of}
\usepackage{hyperref}
\usetheme{Madrid}
\author{Aniket Mishra, Abhishek Bichhawat}
\date{\textit{{[}2025-12-18 Thu]}}
\title{Fall-Through Semantics for Mitigating Timing-Based Side Channel Leaks}
\hypersetup{
 pdfauthor={Aniket Mishra, Abhishek Bichhawat},
 pdftitle={Fall-Through Semantics for Mitigating Timing-Based Side Channel Leaks},
 pdfkeywords={beamer org orgmode},
 pdfsubject={Example of using org to create presentations using the beamer exporter},
 pdfcreator={Emacs 30.2 (Org mode 9.7.11)}, 
 pdflang={English}}
\begin{document}

\maketitle
\begin{frame}{Outline}
\tableofcontents
\end{frame}

\section{Timing-Based Side Channel Leaks}
\label{sec:orgc78f508}
\begin{frame}[label={sec:org57b5422}]{Traditional Cybersecurity}
Traditionally, cybersecurity has dealt with the analysis of \alert{overt} channels.
\begin{center}
\includegraphics[width=.9\linewidth]{./alice_bob.jpg}
\end{center}
\end{frame}
\begin{frame}[label={sec:orgadba5b4}]{What are Side Channels?}
A \emph{side} channel, on the other hand, involves no such direct access. Instead, vulnerabilities are introduced by \alert{side effects} of a program's execution.
\begin{itemize}
\item Power used during the program's execution
\item Sound generated by the machine running the program
\item \alert{Time it takes for the program to execute}
\end{itemize}
\end{frame}
\begin{frame}[label={sec:org335397f},fragile]{An Illustrative Example}
 Let us consider a C program that checks some input against a given password by comparing it by character by character.
\begin{verbatim}
bool matchpwd ( int * input , size_t n ) {
  if ( n != pwd_length ) return false;
  for ( int i = 0; i < n ; i ++) {
    if ( input [ i ] != pwd [ i ]) return false ;
  }
  return true ;
}
\end{verbatim}
\end{frame}
\begin{frame}[label={sec:orgfe475a6}]{An Exploit}
Let's say the password is \alert{101010}, an exploit may look like the following.
\begin{example}[An Attack Trace]\label{sec:org15f0b8e}
000000 \dotfill Rejected in 1\textsuperscript{st} iteration \\
\alert{10}0000 \dotfill Rejected in 3\textsuperscript{rd} iteration \\
\alert{1}10000 \dotfill Rejected in 2\textsuperscript{nd} iteration \\
\alert{1010}00 \dotfill Rejected in 5\textsuperscript{th} iteration \\
\alert{101}100 \dotfill Rejected in 4\textsuperscript{th} iteration \\
\alert{101010} \dotfill Password accepted!
\end{example}
\end{frame}
\begin{frame}[label={sec:org25a84f0}]{The Reality of Timing Side Channels}
Speculative execution and cache side channels are subtler ways that these vulnerabilities can be introduced.
\begin{center}
\includegraphics[width=.9\linewidth]{spooky.png}
\end{center}
\end{frame}
\section{Mitigating Timing-Based Side Channel Leaks}
\label{sec:org24367df}

\section{Fall-Through Semantics for Mitigating Timing-Based Side Channel leaks}
\label{sec:orgfc5ba58}

\section{(Formal) Fall-Through Semantics for Mitigating Timing-Based Side Channel Leaks}
\label{sec:org06fa878}


\section{Methodology}
\label{sec:org7996eff}

\begin{frame}[label={sec:org33fb4d5},fragile]{A simple slide}
 This slide consists of some text with a number of bullet points:

\begin{itemize}
\item the first, very \alert{important}, point!
\item the previous point shows the use of \alert{bold} emphasis which is translated to a \texttt{\textbackslash{}alert\{\}} directive in \LaTeX.
\end{itemize}

The above list could be numbered or any other type of list and may include sub-lists.
\end{frame}
\begin{frame}[label={sec:orgddb4270}]{A more complex slide}
This slide illustrates the use of Beamer blocks.  The following text,
with its own headline, is displayed in a block:
\begin{theorem}[Org mode increases productivity]\label{sec:orgf40c2eb}
\begin{itemize}
\item org mode means not having to remember \LaTeX commands.
\item it is based on ASCII text which is inherently portable.
\item Emacs!
\end{itemize}

\hfill \(\qed\)
\end{theorem}
\end{frame}
\begin{frame}[label={sec:org7509add}]{Two columns}
\begin{columns}
\begin{column}{0.4\columnwidth}
\begin{itemize}
\item this slide consists of two columns
\item the first (left) column has no heading and consists of text
\item the second (right) column has an image and is enclosed in an \alert{example} block
\end{itemize}
\end{column}
\begin{column}{0.5\columnwidth}
\begin{example}[A screenshot]\label{sec:orgbd4a740}
\end{example}
\end{column}
\end{columns}
\end{frame}
\begin{frame}[label={sec:orgf266533},fragile]{Babel}
 This slide shows some code and resulting output using \alert{Babel}.  Note the specification of \texttt{BEAMER\_act} property for the second column.
\begin{columns}
\begin{column}{0.45\columnwidth}
\begin{block}{Octave code}
\end{block}
\end{column}
\begin{column}{0.4\columnwidth}
\begin{block}<2->{The output}
\end{block}
\end{column}
\end{columns}
\end{frame}
\section{Conclusions}
\label{sec:org3324ae1}

\begin{frame}[label={sec:orgb67a76e}]{Summary}
\begin{itemize}
\item org is an incredible tool for time management
\item \alert{but} it is also excellent for writing and for preparing presentations
\item Beamer is a very powerful \LaTeX{} package for presentations
\item the combination is unbeatable!
\end{itemize}
\end{frame}
\end{document}
