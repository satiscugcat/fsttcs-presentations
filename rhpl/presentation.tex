% Created 2025-12-18 Thu 09:36
% Intended LaTeX compiler: pdflatex
\documentclass[presentation]{beamer}
\usepackage[utf8]{inputenc}
\usepackage[T1]{fontenc}
\usepackage{graphicx}
\usepackage{longtable}
\usepackage{wrapfig}
\usepackage{rotating}
\usepackage[normalem]{ulem}
\usepackage{amsmath}
\usepackage{amssymb}
\usepackage{capt-of}
\usepackage{hyperref}
\usetheme{Antibes}
\author{Aniket Mishra}
\date{\textit{{[}2025-12-18 Thu]}}
\title{Mechanical Specification and Verification for Mitigating Timing-based Side Channel Leaks}
\hypersetup{
 pdfauthor={Aniket Mishra},
 pdftitle={Mechanical Specification and Verification for Mitigating Timing-based Side Channel Leaks},
 pdfkeywords={beamer org orgmode},
 pdfsubject={Example of using org to create presentations using the beamer exporter},
 pdfcreator={Emacs 30.2 (Org mode 9.7.11)}, 
 pdflang={English}}
\begin{document}

\maketitle
\begin{frame}{Outline}
\tableofcontents
\end{frame}

\section{Fall-Through Semantics for Mitigating Timing-based Side Channel Leaks}
\label{sec:org213e7c4}
\begin{frame}[label={sec:org8b34fe9},fragile]{An Illustrative Example}
 Let us consider a C program that checks some input against a given password by comparing it by character by character.
\begin{verbatim}
bool matchpwd ( int * input , size_t n ) {
  if ( n != pwd_length ) return false;
  for ( int i = 0; i < n ; i ++) {
    if ( input [ i ] != pwd [ i ]) return false ;
  }
  return true ;
}
\end{verbatim}
\end{frame}
\begin{frame}[label={sec:org0efdb02}]{An Exploit}
Let's say the password is \alert{101010}, an exploit may look like the following.
\begin{example}[An Attack Trace]\label{sec:org233a530}
000000 \dotfill Rejected in 1\textsuperscript{st} iteration \pause \\
\alert{10}0000 \dotfill Rejected in 3\textsuperscript{rd} iteration \pause \\
\alert{1}10000 \dotfill Rejected in 2\textsuperscript{nd} iteration \pause \\
\alert{1010}00 \dotfill Rejected in 5\textsuperscript{th} iteration \pause \\
\alert{101}100 \dotfill Rejected in 4\textsuperscript{th} iteration \pause \\
\alert{101010} \dotfill Password accepted!
\end{example}
\end{frame}
\begin{frame}[label={sec:org63efe2c}]{Threat Model}
Thus, given an adversary that can execute the program under security label \(\ell\). \pause
\begin{block}{Data Security}
The adversary can view the state of the memory, but values that are high relative to \(\ell\) are invisible. \pause
As an example, let us take \(L \sqsubseteq M \sqsubseteq H\) and let us take \(\ell = M\).

Let us say our \emph{\(\mu\)} is \emph{x \(\rightarrow\) true\textsuperscript{L}, y \(\rightarrow\) true\textsuperscript{M}, z \(\rightarrow\) true\textsuperscript{H}}, then the adversary's view of the memory is: \pause

\emph{x \(\rightarrow\) true\textsuperscript{L}, y \(\rightarrow\) true\textsuperscript{M}, z \(\rightarrow\) *}
\pause
\end{block}
\begin{block}{Timing Security}
The adversary has exact knowledge of when and where each memory access takes place during a program's execution.
\end{block}
\end{frame}
\begin{frame}[label={sec:org7729942}]{Expression Semantics}
\begin{center}
\includegraphics[width=.9\linewidth]{./expr.png}
\end{center}
\end{frame}
\begin{frame}[label={sec:org27be4fa}]{Command Semantics}
\begin{block}{The Basics}
\begin{center}
\includegraphics[width=.9\linewidth]{./basics_c.png}
\end{center}
\pause
\end{block}
\begin{block}{IF-HIGH}
\begin{center}
\includegraphics[width=.9\linewidth]{./ifhigh_c.png}
\end{center}
\end{block}
\end{frame}
\begin{frame}[label={sec:org79b2daf}]{Debranch Semantics: The Basics}
\begin{center}
\includegraphics[width=.9\linewidth]{./basics_d.png}
\end{center} 
\end{frame}
\begin{frame}[label={sec:orgfe1cfbc}]{Debranch Semantics: IFs}
\begin{center}
\includegraphics[width=.9\linewidth]{./ifs.png}
\end{center}
\end{frame}
\section{Interactive Theorem Proving - What's the Fuss About?}
\label{sec:org51f1ef3}
\begin{frame}[label={sec:org83a3504}]{ITPs}
\alert{Interactive Theorem Provers} have been around for quite a while. However, they have been facing a lot of very \emph{recent} adoption.
\begin{itemize}
\item LEAN4 (2013)
\item FStar (2011)
\item Agda (1999)
\item \alert{Coq/Rocq (1989/2025)}
\item Isabelle (1986)
\item Automath (1967)
\end{itemize}
\end{frame}
\begin{frame}[label={sec:org65011d2}]{Expressing Mathematical Structures}
\begin{center}
\includegraphics[width=.9\linewidth]{./simple.png}
\end{center}
\end{frame}
\begin{frame}[label={sec:org09f1bdc}]{Why Bother?}
The most important thing that we get from these systems is \alert{trust}.
\begin{itemize}
\item \alert{Manual verification} can be error-prone and time-consuming.
\item With a theorem prover, the implementation of the core system is comparitively very \alert{small}.
\item With this base, it is much easier to \alert{trust results} (although there are caveats we will discuss later).
\end{itemize}
\end{frame}
\section{Mechanical Specification and Verification for Mitigating Timing-based Side Channel Leaks}
\label{sec:org5f3507b}
\begin{frame}[label={sec:org930d4ca}]{Expressing Grammars}
\begin{center}
\includegraphics[width=.9\linewidth]{./grammar.png}
\end{center}
\end{frame}
\begin{frame}[label={sec:org9400713}]{Expressing Semantics}
\begin{center}
\includegraphics[width=.9\linewidth]{./semantics.png}
\end{center}
\end{frame}
\begin{frame}[label={sec:org650d95a}]{Expressing Invariants}
\begin{center}
\includegraphics[width=.9\linewidth]{./invariant.png}
\end{center}
\end{frame}
\section{Interactive Theorem Proving - An Undegraduate Perspective}
\label{sec:org656c51f}
\begin{frame}[label={sec:orgfb6a9df}]{Why Did I Bother?}
\begin{itemize}
\item I started working on this project during the first semester of my second year. I was taking Discrete Math \alert{at the same time}! \pause
\item During December, I spent some time in IIT Delhi with Prof. Vaishnavi Sundararajan and started properly working with theorem provers. \pause
\item After this, I got extremely anxious about my \alert{proofs and theorems being incorrect}.
\end{itemize}
\end{frame}
\begin{frame}[label={sec:orgeafb7b8}]{Learning to write Rocq vs Learning to write Proofs}
\begin{block}{Learning Rocq}
\begin{itemize}
\item \alert{Well documented} and maintained by a team of highly qualified researchers/engineers. \pause
\item Wonderful widely available resources for learning \alert{(Software Foundations)}. \pause
\item At the end of the day, it is a programming language! Possibility of \alert{transfer}. \pause
\end{itemize}
\end{block}
\begin{block}{Learning Proofs}
\begin{itemize}
\item Universally accepted resources do not exist (as far as I know). \pause
\item Discrete Math courses often focus on \alert{already existing/well-studied structures}. \pause But (from what I know), not as much work in \alert{defining and using your own}. \pause
\item Learning by example is difficult: "The proof is trivial", or "The proof follows simply from induction on \emph{x}".
\end{itemize}
\end{block}
\end{frame}
\begin{frame}[label={sec:org11cd5ed}]{Intuition}
Intuition plays a big role in proofs. \pause Take for example the pumping lemma for context free grammars. \pause
\begin{center}
\includegraphics[width=.9\linewidth]{./sipser.png}
\end{center}
\end{frame}
\begin{frame}[label={sec:org70635b9}]{Intuition in ITPs}
\begin{itemize}
\item There is \alert{almost no room} for intuition in ITPs. \pause
\item This is bad. Spending time on things obvious by intuition, may hamper the \alert{non-obvious} things. \pause
\item This is good. It allows for stronger guarantees. \pause \alert{As a student, by default, your intuition is non-existent or bad.}
\end{itemize}
\end{frame}
\begin{frame}[label={sec:orga9045d9}]{Costs and Caveats}
\pause
\begin{itemize}
\item Error in \alert{specification}. \pause
\item \alert{Bugs} in prover-software. Possibility of \alert{breaking updates}. \pause
\item Benefits from \alert{abstractions}. \pause
\item Strictness can cause difficulty in \alert{iteration}.
\end{itemize}
\end{frame}
\begin{frame}[label={sec:orgf528d51}]{Concluding}
That's all! Thank you for attending my talk. I am part of a student club called \alert{,$\backslash$\ AMBDA.} at IIT Gandhinagar where we like to work on interesting things in PLT along with organising talks to cultivate interest in PLT. Let me know if you are interested in knowing more! You can also contact me at \url{mailto:aniket.mishra@iitgn.ac.in}.
\end{frame}
\end{document}
